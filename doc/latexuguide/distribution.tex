\section{DISTRIBUTION}
\label{sec:track}

The \texttt{DISTRIBUTION} creates a distribution based on the last TWISS.  


\madbox{
xxxxxxx\= \kill
TRACK, \>HORIZONTAL=string, VERTICAL=string, LONGITUDINAL=string; \\
       \>CUTSIG_H=\{real, real \}, CUTSIG_V=\{real, real \}, CUTSIG_L=\{real, real \}, \\
       \>NPART=real, ORBIT=LOGICAL, FILE=filename;
}

The attributes of the DISTRIBUTION command are:

\begin{madlist}
  \ttitem{DELTAP} relative momentum offset for reference closed orbit (switched
  off for \texttt{ONEPASS}) \\  
  Defining a non-zero \texttt{DELTAP} results in a change of the beam
  momentum/energy without changing the magnetic properties in the
  sequence, which leads to an off-momentum closed orbit different from
  the on-momentum reference orbit. Particle coordinates are then given
  with respect to this new closed orbit, unless the option
  \texttt{ONEPASS=true} is used! \\  
  (Default:~0.0)

  \ttitem{ONEPASS} flag to ensure that no closed orbit search is done,
  which also means that no stability test is done. This is always the
  case for transfer lines, but this option can also be enabled for
  multi-turn tracking of a circular machine. \texttt{ONEPASS=true} does
  \textbf{NOT} restrict tracking to a single turn. \\
  With \texttt{ONEPASS=true}, the particle coordinates are specified with
  respect to the reference orbit. \\  
  With \texttt{ONEPASS=false}, the closed orbit is calculated and the particle
  coordinates are given with respect to the closed orbit coordinates.\\
  This flag affects the behavior of the \hyperref[sec:option]{\texttt{BBORBIT}} flag. \\
  The name of this attribute is misleading but was kept for backwards
  compatibility.  \\ 
  (Default:~false)

  \ttitem{DAMP} flag to introduce synchrotron damping (needs RF cavity
  and flag \texttt{RADIATE} in the \texttt{BEAM} command). \\ (Default:~false)

  \ttitem{QUANTUM} flag to introduce quantum excitation via random
  number generator and tables look-up (\texttt{SYNRAD} $=1$, see ref. \cite{roy1990}) or polynomial
  interpolation (\texttt{SYNRAD} $=2$, see ref. \cite{hbu2007}) for photon emission.
  The choice of the generator can be selected via the command
  \hyperref[sec:option]{\texttt{OPTION}} attribute \texttt{SYNRAD}. \\ (Default:~2)

  \ttitem{SEED} If \texttt{QUANTUM} is true, it selects a particular sequence of random values. 
A \texttt{SEED} value is an integer in the range [0...999999999] (default:
123456789). Note that the seed set with this command is shared with the \hyperref[sec:coption]{\texttt{COPTION}} and \hyperref[sec:coption]{\texttt{EOPTION}} commands. See also: \hyperref[subsubsec:random]{Random Values}.

  \ttitem{DUMP} flag to write the particle coordinates in files, whose
  names are generated automatically. \\ (Default:~false)

  \ttitem{APERTURE} a logical flag to trigger aperture check at the entrance 
  of each element (except \texttt{DRIFT}s). A particle is lost from the table of 
  tracked particles if its position lies outside the aperture of the current 
  element at the entrance of this element. \\ 
  (Default:~false) \\
  
  The \hyperref[chap:aperture]{\texttt{APERTYPE}} and 
  \hyperref[chap:aperture]{\texttt{APERTURE}} information of each element 
  in the sequence is used to assess the particle loss. 
  However \texttt{TRACK} only takes into account the predefined aperture 
  types listed in table \ref{table:apertype}
  \\
  
  Note that if no aperture information was specified for an element, 
  the following procedure still takes place:
  \\
  $\rightarrow$ No aperture definition for element $\rightarrow$ 
  Default apertype/aperture assigned (currently this is   
  \texttt{APERTYPE=circle, APERTURE=\{0\}}) 
  \\ $\rightarrow$  
  If tracking with \texttt{APERTURE} is used and an
  element with \texttt{APERTYPE=circle} AND \texttt{APERTURE=\{0\}}  
  is encountered, then the first value of the \texttt{MAXAPER} vector
  is assigned as the circle's radius (no permanent assignment!). 
  See option \hyperref[sec:run]{\texttt{MAXAPER}} for the default values. 
  \\ $\Rightarrow$
  Hence even if no aperture information is specified by the user for
  certain elements, default values will be used! 


  \ttitem{ONETABLE} flag to write all particle coordinates in a single
  file instead of one file per particle. \\ (Default:~false)

  \ttitem{RECLOSS} flag to create in memory a table named "trackloss"
  containing the coordinates of lost particles.\\
  (Default:~false) \\
  Traditionally, when a particle is lost on the aperture, this information
  is written to stdout. To allow more flexible tracking studies, the
  coordinates of lost particles and additional information can also be
  saved in a table in memory. Usually one would save this table to a
  file using the \texttt{WRITE} command after the tracking run has
  finished. The following information is available in the TFS table
  "trackloss":          
  \begin{itemize}
  \item Particle ID (number)
  \item Turn number
  \item Particle coordinates (x,px,y,py,t,pt)
  \item Longitudinal position in the machine (s)
  \item Beam energy
  \item Element name, where the particle is lost
  \end{itemize}

  \ttitem{FILE} name for the track table. The default name is different
  depending on the value of the \texttt{ONETABLE} attribute. \\ 
  (Default: "track" if \texttt{ONETABLE=true}, "trackone" if \texttt{ONETABLE=false})

  \ttitem{UPDATE} flag to trigger parameter update per turn. \\  
  (Default:~false) \\
  Specifying \texttt{UPDATE=true} gives access to the following additions:   
  \begin{madlist}
    \ttitem{tr\$turni} this special variable contains the turn number;
    it can be used in expressions like \texttt{KICK := SIN(tr\$turni)} and is
    updated at each turn during tracking.     
    \ttitem{tr\$macro}  this special macro can be
    user-defined and is executed/updated at each turn, during tracking.
    A macro structure is necessary to provide for table access.
    \textsl{e.g.} \\ 
    \texttt{
      tr\$macro(turn): macro=\{ \\
      commands that can depend on the turnnumber;\\
      \};
    } 
  \end{madlist}

\end{madlist}

\textbf{Remarks}\\
\emph{IMPORTANT:} If an RF cavity has a non-zero voltage, synchrotron
oscillations are automatically included. If tracking with constant
momentum is desired, then the voltage of the RF cavities has to be set
to zero. If an RF cavity has a no zero voltage and \texttt{DELTAP} is non zero, 
tracking is done with synchrotron oscillations around an off-momentum
closed orbit.