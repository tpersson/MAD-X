\section{DISTRIBUTION}
\label{sec:track}

The \texttt{DISTRIBUTION} creates a distribution based on the last TWISS.  


\madbox{
xxxxxxx\= \kill
TRACK, \>HORIZONTAL=string, VERTICAL=string, LONGITUDINAL=string; \\
       \>CUTSIG_H=\{real, real \}, CUTSIG_V=\{real, real \}, CUTSIG_L=\{real, real \}, \\
       \>NPART=real, TABLE=string;
}

The attributes of the DISTRIBUTION command are:

\begin{madlist}  
\ttitem{HORIZONTAL,VERTICAL,LONGITUDINAL} This defines the planes.
	\begin{madlist}  
    \ttitem{GAUSS} This gives a gaussian distribution with the emittance defined
    by the beam command. The uncopuled distribution generated will follow:
    $e_x = \sqrt{<x^2><p_x^2>-<xp_x>^2}$.  
    \ttitem{UNIFORM} A uniform distribution in phase space for the selected plane.
    The intervall is given in number of beam emittances for the particular plane. 
    \ttitem{FIXED} Assigns the same value for all the particles for that plane. 
    The value is set to $2J$ in single particle emittance which gives that the particle
    tracked over many turns will again give the emittance defined by the beam command. 
  \end{madlist}
\ttitem{CUTSIG_X} Takes 2 values. It cuts any values below the first value and any value above the second. 
The units of the cuts are given in sigma. For the UNIFORM distribution this defines the start and stop value.   
\ttitem{NPART} Number of particle generated
\ttitem{TABLE} The name of the table where the distribution is stored.
\end{madlist}

\textbf{Remarks}\\
\emph{IMPORTANT:} 